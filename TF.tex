\documentclass[12pt]{article}
\usepackage[utf8x]{inputenc}
\usepackage{ifluatex}
\usepackage{stata}
\usepackage{graphicx}
\usepackage[spanish]{babel}
\usepackage{amsmath}
\usepackage{amsfonts}
\usepackage{amssymb}
\usepackage{graphicx}
\usepackage[left = 2.54cm, right=2.54cm, top=2.54cm, bottom=2.00cm]{geometry}
\usepackage{caption}
\usepackage{subcaption}
\usepackage{verbatim} % comentarios
\usepackage[paper=portrait,pagesize]{typearea}
\usepackage{apacite}
\usepackage{adjustbox}
\usepackage{blindtext}
\usepackage{titlesec}
\usepackage{lscape}
\usepackage[flushleft, para]{threeparttable}
\title{EVALUACIÓN DE IMPACTO PARA LA GESTIÓN PÚBLICA }
\author{Juan Manuel Rivas Castillo \\
\large Examen Final}
\date{08 de agosto de 2023}




\begin{document}

%\maketitle

\tableofcontents\newpage



\section{Indicaciones:}

– El trabajo consiste en realizar dos de los tres ejercicios propuestos en este documento. 

\vspace{0.5cm}

– La entrega consiste en el do file (STATA) comentado. También debe entregar los resultados (tablas, gráficos, etc.) explicados en un archivo Word o PDF. 

\vspace{0.5cm}

– El trabajo se puede realizar en grupos de máximo 4 integrantes. 


\vspace{0.5cm}

\section{DIFERENCIAS EN DIFERENCIAS}

El programa de transferencias condicionadas PROGRESA, conocido también como Oportunidades o Programa de Desarrollo Humano Oportunidades, es una iniciativa social implementada en México con el propósito de combatir la pobreza y promover el desarrollo humano en comunidades vulnerables del país. 

\vspace{0.5cm}

El objetivo principal de PROGRESA es reducir la pobreza extrema y mejorar las condiciones de vida de las familias más desfavorecidas. Para lograrlo, el programa otorga transferencias económicas mensuales a las familias seleccionadas, siempre y cuando cumplan con ciertos requisitos y condiciones establecidas. Los beneficiarios son familias que viven en áreas rurales o urbanas con altos índices de pobreza y marginalidad, y que cumplen con los criterios definidos por el programa. 

\vspace{0.5cm}

Entre las condiciones se encuentran aquellas relacionadas con la asistencia escolar de los niños. Los beneficiarios deben asegurar que sus hijos asistan regularmente a la escuela y que se mantengan inscritos para recibir el apoyo económico. De esta manera, PROGRESA busca fomentar la educación como una herramienta para romper el ciclo de pobreza. Además de la educación, el programa también pone énfasis en la salud y nutrición de las familias; los beneficiarios deben participar en programas de salud que incluyen chequeos médicos regulares, vacunación de los niños y acceso a servicios básicos de atención médica. 

\vspace{0.5cm}

El programa PROGRESA se somete a evaluaciones periódicas para medir su efectividad y asegurar el cumplimiento de los objetivos establecidos. Esto permite ajustar y mejorar el programa con el tiempo, buscando una mayor eficacia en el combate contra la pobreza y la promoción del desarrollo humano en México. 

\vspace{0.5cm}

La base de datos panel\_progresa\_V1.dta contiene una submuestra de niños y jóvenes con la cual se puede medir el efecto que tiene la transferencia monetaria sobre la tasa de asistencia escolar. 

\vspace{0.5cm}

El programa comenzó a intervenir a los beneficiarios en junio de 1998. Hizo 2 levantamientos de información previos a la intervención (octubre de 1997 y marzo de 1998) y 3 levantamientos posteriores a la intervención (octubre de 1998, mayo de 1999 y noviembre de 1999). El tratamiento fue asignado de forma aleatoria a nivel de
centros poblados o localidades. No obstante, solamente una fracción de los habitantes de una localidad participante se terminó beneficiando. 

\vspace{0.5cm}
El propósito de este ejercicio es estimar el efecto del programa sobre la tasa de asistencia escolar mediante el estimador de diferencias en diferencias. Para lograr esto se necesitan 3 variables: una variable de tiempo post (después/antes de la intervención), una variable de tratamiento (si el hogar fue tratado o no) y un término de interacción de ambas, el cual es nuestro parámetro de interés. 

\vspace{0.5cm}

a) Organice (comando sort) la base en función de las variables "folio" (identificado de hogar), 'ID' (identificador del individuo dentro de su hogar) y 'periodo'. Genere un identificador para cada individuo juntando las variables ID y folio. 
\begin{stlog}. cd "C:\\Users\\Juan Rivas\\Documents\\Drive\\CIES"
C:\\Users\\Juan Rivas\\Documents\\Drive\\CIES
{\smallskip}
. clear all 
{\smallskip}
. notes: Cargamos la base de datos
{\smallskip}
. use panel_progresa_V1.dta, clear 
{\smallskip}
. notes: Listamos los primeros 10 datos de folio e ID
{\smallskip}
. list  folio ID periodo in 1/10
{\smallskip}
     {\TLC}\HLI{29}{\TRC}
     {\VBAR} folio   ID          periodo {\VBAR}
     {\LFTT}\HLI{29}{\RGTT}
  1. {\VBAR} 11038    3     Octubre 1997 {\VBAR}
  2. {\VBAR} 11038    3   Noviembre 1999 {\VBAR}
  3. {\VBAR} 11038    3        Mayo 1999 {\VBAR}
  4. {\VBAR} 11038    3       Marzo 1998 {\VBAR}
  5. {\VBAR} 11038    3     Octubre 1998 {\VBAR}
     {\LFTT}\HLI{29}{\RGTT}
  6. {\VBAR} 11038    4     Octubre 1997 {\VBAR}
  7. {\VBAR} 11038    4       Marzo 1998 {\VBAR}
  8. {\VBAR} 11038    4        Mayo 1999 {\VBAR}
  9. {\VBAR} 11038    4     Octubre 1998 {\VBAR}
 10. {\VBAR} 11038    4   Noviembre 1999 {\VBAR}
     {\BLC}\HLI{29}{\BRC}
{\smallskip}
. notes: Construímos una variable local para almacenar las variables a ordenar
{\smallskip}
. local vars "folio ID periodo"
{\smallskip}
. sort `vars'
{\smallskip}
. notes: Observamos las primeras 10 observaciones de las variables de interés
{\smallskip}
. list  folio ID periodo in 1/10
{\smallskip}
     {\TLC}\HLI{29}{\TRC}
     {\VBAR} folio   ID          periodo {\VBAR}
     {\LFTT}\HLI{29}{\RGTT}
  1. {\VBAR} 11038    3     Octubre 1997 {\VBAR}
  2. {\VBAR} 11038    3       Marzo 1998 {\VBAR}
  3. {\VBAR} 11038    3     Octubre 1998 {\VBAR}
  4. {\VBAR} 11038    3        Mayo 1999 {\VBAR}
  5. {\VBAR} 11038    3   Noviembre 1999 {\VBAR}
     {\LFTT}\HLI{29}{\RGTT}
  6. {\VBAR} 11038    4     Octubre 1997 {\VBAR}
  7. {\VBAR} 11038    4       Marzo 1998 {\VBAR}
  8. {\VBAR} 11038    4     Octubre 1998 {\VBAR}
  9. {\VBAR} 11038    4        Mayo 1999 {\VBAR}
 10. {\VBAR} 11038    4   Noviembre 1999 {\VBAR}
     {\BLC}\HLI{29}{\BRC}
{\smallskip}
. notes: Creamos el identificador de grupos
{\smallskip}
. egen Id = group(folio ID)
{\smallskip}
. notes: Generamos el número de observaciones por individuos
{\smallskip}
. label var Id "Identidicador por individuos"
{\smallskip}
. bys folio ID (periodo): gen Periodo = _n
{\smallskip}
. notes: Volvemos a listar la base con las nuevas variables
{\smallskip}
. list  Id Periodo in 1/10
{\smallskip}
     {\TLC}\HLI{14}{\TRC}
     {\VBAR} Id   Periodo {\VBAR}
     {\LFTT}\HLI{14}{\RGTT}
  1. {\VBAR}  1         1 {\VBAR}
  2. {\VBAR}  1         2 {\VBAR}
  3. {\VBAR}  1         3 {\VBAR}
  4. {\VBAR}  1         4 {\VBAR}
  5. {\VBAR}  1         5 {\VBAR}
     {\LFTT}\HLI{14}{\RGTT}
  6. {\VBAR}  2         1 {\VBAR}
  7. {\VBAR}  2         2 {\VBAR}
  8. {\VBAR}  2         3 {\VBAR}
  9. {\VBAR}  2         4 {\VBAR}
 10. {\VBAR}  2         5 {\VBAR}
     {\BLC}\HLI{14}{\BRC}
{\smallskip}
\end{stlog}

\vspace{0.5cm}
b) Genere una variable dummy para cada ronda de encuesta. Es decir, convierta la variable categórica que muestra la base a una variable binaria para cada ronda de encuesta. 

\begin{stlog}. 
. notes: Nos generamos una dummy por cada periodo de la encuesta
{\smallskip}
. tab periodo, nol
{\smallskip}
   Fecha de {\VBAR}
   encuesta {\VBAR}      Freq.     Percent        Cum.
\HLI{12}{\PLUS}\HLI{35}
          1 {\VBAR}     21,113       20.00       20.00
          2 {\VBAR}     21,113       20.00       40.00
          3 {\VBAR}     21,113       20.00       60.00
          4 {\VBAR}     21,113       20.00       80.00
          5 {\VBAR}     21,113       20.00      100.00
\HLI{12}{\PLUS}\HLI{35}
      Total {\VBAR}    105,565      100.00
{\smallskip}
. forvalues i=1/5 {\lbr}
  2.     g enc`i'=0
  3.         replace enc`i' = 1 if periodo==`i'
  4. {\rbr}
(21,113 real changes made)
(21,113 real changes made)
(21,113 real changes made)
(21,113 real changes made)
(21,113 real changes made)
{\smallskip}
\end{stlog}



c) Genere las variables post, tratamiento e interacción. 

\begin{stlog}. **** Generamos variable dummmy cuando el tratamiento comienza. El programa comenzó a intervenir 
> a los beneficiarios en junio de 1998
. gen time = (periodo>=3) \& !missing(periodo) 
{\smallskip}
. rename part_ treated
{\smallskip}
. **** Creamos la interaccion entre ambos
. gen did = time*treated  
{\smallskip}
. **** Le indicamos a stata que trabajamos con un panel
. xtset Id Periodo 
{\smallskip}
Panel variable: Id (strongly balanced)
 Time variable: Periodo, 1 to 5
         Delta: 1 unit
{\smallskip}
. 
\end{stlog}


d) Genere una dummy si el hogar es considerado pobre en la primera línea de base (octubre de 1997). Hacer esto por "folio" usando el comando bysort. Es decir, si al menos 1 individuo es considerado pobre, todos los individuos de ese folio también sean considerados pobres. 

\begin{stlog}. gen poLB = 0
{\smallskip}
. bys folio: replace poLB = 1 if pobre == 1 \& periodo==1
(14,748 real changes made)
{\smallskip}
. egen polb= max(poLB > 0), by(folio)
{\smallskip}
. tab polb
{\smallskip}
       polb {\VBAR}      Freq.     Percent        Cum.
\HLI{12}{\PLUS}\HLI{35}
          0 {\VBAR}     31,825       30.15       30.15
          1 {\VBAR}     73,740       69.85      100.00
\HLI{12}{\PLUS}\HLI{35}
      Total {\VBAR}    105,565      100.00
{\smallskip}
\end{stlog}


e) Como la población objetivo del programa son los pobres, quédese solo con los individuos que se consideran pobres en la primera línea de base (octubre de 1997). 

\begin{stlog}. keep if polb!=0
(31,825 observations deleted)
{\smallskip}
\end{stlog}

f) Estime el efecto medio del tratamiento aplicando el método de doble diferencia. No olvide mostrar el valor de la 1era diferencia y de la 2da diferencia. 

\begin{stlog}. xtset Id Periodo
{\smallskip}
Panel variable: Id (strongly balanced)
 Time variable: Periodo, 1 to 5
         Delta: 1 unit
{\smallskip}
. * Con reg
. reg everschool time treated did
{\smallskip}
      Source {\VBAR}       SS           df       MS      Number of obs   =    73,291
\HLI{13}{\PLUS}\HLI{34}   F(3, 73287)     =     70.07
       Model {\VBAR}  5.51041576         3  1.83680525   Prob > F        =    0.0000
    Residual {\VBAR}  1920.99871    73,287  .026211998   R-squared       =    0.0029
\HLI{13}{\PLUS}\HLI{34}   Adj R-squared   =    0.0028
       Total {\VBAR}  1926.50912    73,290  .026286112   Root MSE        =     .1619
{\smallskip}
\HLI{13}{\TOPT}\HLI{64}
  everschool {\VBAR} Coefficient  Std. err.      t    P>|t|     [95\% conf. interval]
\HLI{13}{\PLUS}\HLI{64}
        time {\VBAR}   .0205864   .0019263    10.69   0.000     .0168108    .0243619
     treated {\VBAR}   .0123945   .0019265     6.43   0.000     .0086186    .0161705
         did {\VBAR}  -.0084079   .0024894    -3.38   0.001    -.0132871   -.0035287
       _cons {\VBAR}   .9562489   .0014908   641.43   0.000      .953327    .9591709
\HLI{13}{\BOTT}\HLI{64}
{\smallskip}
. 
. * Con xtreg
. xtreg everschool time treated did, fe 
note: {\bftt{treated}} omitted because of collinearity.
{\smallskip}
Fixed-effects (within) regression               Number of obs     =     73,291
Group variable: Id                              Number of groups  =     14,748
{\smallskip}
R-squared:                                      Obs per group:
     Within  = 0.0062                                         min =          2
     Between = 0.0002                                         avg =        5.0
     Overall = 0.0017                                         max =          5
{\smallskip}
                                                F(2,58541)        =     183.58
corr(u_i, Xb) = -0.0105                         Prob > F          =     0.0000
{\smallskip}
\HLI{13}{\TOPT}\HLI{64}
  everschool {\VBAR} Coefficient  Std. err.      t    P>|t|     [95\% conf. interval]
\HLI{13}{\PLUS}\HLI{64}
        time {\VBAR}   .0196548   .0012939    15.19   0.000     .0171188    .0221907
     treated {\VBAR}          0  (omitted)
         did {\VBAR}   -.007285   .0016721    -4.36   0.000    -.0105624   -.0040077
       _cons {\VBAR}   .9638256   .0006335  1521.49   0.000      .962584    .9650673
\HLI{13}{\PLUS}\HLI{64}
     sigma_u {\VBAR}  .13072244
     sigma_e {\VBAR}  .10841536
         rho {\VBAR}  .59247658   (fraction of variance due to u_i)
\HLI{13}{\BOTT}\HLI{64}
F test that all u_i=0: F(14747, 58541) = 7.12                Prob > F = 0.0000
{\smallskip}
. 
. * Forma alternativa:
. xtreg everschool time\#\#treated, fe 
note: {\bftt{1.treated}} omitted because of collinearity.
{\smallskip}
Fixed-effects (within) regression               Number of obs     =     73,291
Group variable: Id                              Number of groups  =     14,748
{\smallskip}
R-squared:                                      Obs per group:
     Within  = 0.0062                                         min =          2
     Between = 0.0002                                         avg =        5.0
     Overall = 0.0017                                         max =          5
{\smallskip}
                                                F(2,58541)        =     183.58
corr(u_i, Xb) = -0.0105                         Prob > F          =     0.0000
{\smallskip}
\HLI{13}{\TOPT}\HLI{64}
  everschool {\VBAR} Coefficient  Std. err.      t    P>|t|     [95\% conf. interval]
\HLI{13}{\PLUS}\HLI{64}
      1.time {\VBAR}   .0196548   .0012939    15.19   0.000     .0171188    .0221907
             {\VBAR}
     treated {\VBAR}
  Participa  {\VBAR}          0  (omitted)
             {\VBAR}
time\#treated {\VBAR}
1\#Participa  {\VBAR}   -.007285   .0016721    -4.36   0.000    -.0105624   -.0040077
             {\VBAR}
       _cons {\VBAR}   .9638256   .0006335  1521.49   0.000      .962584    .9650673
\HLI{13}{\PLUS}\HLI{64}
     sigma_u {\VBAR}  .13072244
     sigma_e {\VBAR}  .10841536
         rho {\VBAR}  .59247658   (fraction of variance due to u_i)
\HLI{13}{\BOTT}\HLI{64}
F test that all u_i=0: F(14747, 58541) = 7.12                Prob > F = 0.0000
{\smallskip}
. 
. 
. **** Paquete diff: nos hace todo esto sin escribir las regresiones
. 
. *ssc install diff // Lo instalamos 
. 
. diff everschool, t(treated) p(time)
{\smallskip}
{\bftt{{\underbar{DIFFERENCE-IN-DIFFERENCES ESTIMATION RESULTS}}}}
Number of observations in the DIFF-IN-DIFF: 73291
            Before         After    
   Control: 11794          17613       29407
   Treated: 17604          26280       43884
            29398          43893
\HLI{56}
 Outcome var.   {\VBAR} evers{\tytilde}l {\VBAR} S. Err. {\VBAR}   |t|   {\VBAR}  P>|t|
\HLI{16}{\PLUS}\HLI{9}{\PLUS}\HLI{9}{\PLUS}\HLI{9}{\PLUS}\HLI{9}
Before          {\VBAR}         {\VBAR}         {\VBAR}         {\VBAR} 
   Control      {\VBAR} 0.956   {\VBAR}         {\VBAR}         {\VBAR} 
   Treated      {\VBAR} 0.969   {\VBAR}         {\VBAR}         {\VBAR} 
   Diff (T-C)   {\VBAR} 0.012   {\VBAR} 0.002   {\VBAR} 6.43    {\VBAR} 0.000***
After           {\VBAR}         {\VBAR}         {\VBAR}         {\VBAR} 
   Control      {\VBAR} 0.977   {\VBAR}         {\VBAR}         {\VBAR} 
   Treated      {\VBAR} 0.981   {\VBAR}         {\VBAR}         {\VBAR} 
   Diff (T-C)   {\VBAR} 0.004   {\VBAR} 0.002   {\VBAR} 2.53    {\VBAR} 0.011**
                {\VBAR}         {\VBAR}         {\VBAR}         {\VBAR} 
Diff-in-Diff    {\VBAR} -0.008  {\VBAR} 0.002   {\VBAR} 3.38    {\VBAR} 0.001***
\HLI{56}
R-square:    0.00
* Means and Standard Errors are estimated by linear regression
**Inference: *** p<0.01; ** p<0.05; * p<0.1
{\smallskip}
\end{stlog}


g) Proponga y aplique mejoras a la estimación realizada: puede incluir covariables, clusterizar errores, corregir heterocedasticidad, etc. 

\vspace{0.5cm}

\begin{stlog}. * DiD sin variables de control
. diff everschool, t(treated) p(time)
{\smallskip}
{\bftt{{\underbar{DIFFERENCE-IN-DIFFERENCES ESTIMATION RESULTS}}}}
Number of observations in the DIFF-IN-DIFF: 73291
            Before         After    
   Control: 11794          17613       29407
   Treated: 17604          26280       43884
            29398          43893
\HLI{56}
 Outcome var.   {\VBAR} evers{\tytilde}l {\VBAR} S. Err. {\VBAR}   |t|   {\VBAR}  P>|t|
\HLI{16}{\PLUS}\HLI{9}{\PLUS}\HLI{9}{\PLUS}\HLI{9}{\PLUS}\HLI{9}
Before          {\VBAR}         {\VBAR}         {\VBAR}         {\VBAR} 
   Control      {\VBAR} 0.956   {\VBAR}         {\VBAR}         {\VBAR} 
   Treated      {\VBAR} 0.969   {\VBAR}         {\VBAR}         {\VBAR} 
   Diff (T-C)   {\VBAR} 0.012   {\VBAR} 0.002   {\VBAR} 6.43    {\VBAR} 0.000***
After           {\VBAR}         {\VBAR}         {\VBAR}         {\VBAR} 
   Control      {\VBAR} 0.977   {\VBAR}         {\VBAR}         {\VBAR} 
   Treated      {\VBAR} 0.981   {\VBAR}         {\VBAR}         {\VBAR} 
   Diff (T-C)   {\VBAR} 0.004   {\VBAR} 0.002   {\VBAR} 2.53    {\VBAR} 0.011**
                {\VBAR}         {\VBAR}         {\VBAR}         {\VBAR} 
Diff-in-Diff    {\VBAR} -0.008  {\VBAR} 0.002   {\VBAR} 3.38    {\VBAR} 0.001***
\HLI{56}
R-square:    0.00
* Means and Standard Errors are estimated by linear regression
**Inference: *** p<0.01; ** p<0.05; * p<0.1
{\smallskip}
. 
. * Bootstrapped std. err.:
. diff everschool, t(treated) p(time) bs rep(50)
(running {\bftt{regress}} on estimation sample)
{\smallskip}
Bootstrap replications (50)
\HLI{4}{\PLUS}\HLI{3} 1 \HLI{3}{\PLUS}\HLI{3} 2 \HLI{3}{\PLUS}\HLI{3} 3 \HLI{3}{\PLUS}\HLI{3} 4 \HLI{3}{\PLUS}\HLI{3} 5 
..................................................    50
{\smallskip}
{\bftt{{\underbar{DIFFERENCE-IN-DIFFERENCES ESTIMATION RESULTS}}}}
Number of observations in the DIFF-IN-DIFF: 73291
            Before         After    
   Control: 11794          17613       29407
   Treated: 17604          26280       43884
            29398          43893
Bootstrapped Standard Errors
{\smallskip}
\HLI{56}
 Outcome var.   {\VBAR} evers{\tytilde}l {\VBAR} S. Err. {\VBAR}   |t|   {\VBAR}  P>|t|
\HLI{16}{\PLUS}\HLI{9}{\PLUS}\HLI{9}{\PLUS}\HLI{9}{\PLUS}\HLI{9}
Before          {\VBAR}         {\VBAR}         {\VBAR}         {\VBAR} 
   Control      {\VBAR} 0.956   {\VBAR}         {\VBAR}         {\VBAR} 
   Treated      {\VBAR} 0.969   {\VBAR}         {\VBAR}         {\VBAR} 
   Diff (T-C)   {\VBAR} 0.012   {\VBAR} 0.002   {\VBAR} 5.11    {\VBAR} 0.000***
After           {\VBAR}         {\VBAR}         {\VBAR}         {\VBAR} 
   Control      {\VBAR} 0.977   {\VBAR}         {\VBAR}         {\VBAR} 
   Treated      {\VBAR} 0.981   {\VBAR}         {\VBAR}         {\VBAR} 
   Diff (T-C)   {\VBAR} 0.004   {\VBAR} 0.001   {\VBAR} 2.91    {\VBAR} 0.004***
                {\VBAR}         {\VBAR}         {\VBAR}         {\VBAR} 
Diff-in-Diff    {\VBAR} -0.008  {\VBAR} 0.003   {\VBAR} 2.99    {\VBAR} 0.003***
\HLI{56}
R-square:    0.00
* Means and Standard Errors are estimated by linear regression
**Inference: *** p<0.01; ** p<0.05; * p<0.1
{\smallskip}
. 
. * DiD con variables de control (covariates)
. diff everschool, t(treated) p(time) cov( edad hhsize educfather educmother nodad nomom  sex)
{\bftt{{\underbar{DIFFERENCE-IN-DIFFERENCES WITH COVARIATES}}}}
{\smallskip}
{\bftt{{\underbar{DIFFERENCE-IN-DIFFERENCES ESTIMATION RESULTS}}}}
Number of observations in the DIFF-IN-DIFF: 58679
            Before         After    
   Control: 9363           13986       23349
   Treated: 14162          21168       35330
            23525          35154
\HLI{56}
 Outcome var.   {\VBAR} evers{\tytilde}l {\VBAR} S. Err. {\VBAR}   |t|   {\VBAR}  P>|t|
\HLI{16}{\PLUS}\HLI{9}{\PLUS}\HLI{9}{\PLUS}\HLI{9}{\PLUS}\HLI{9}
Before          {\VBAR}         {\VBAR}         {\VBAR}         {\VBAR} 
   Control      {\VBAR} 0.900   {\VBAR}         {\VBAR}         {\VBAR} 
   Treated      {\VBAR} 0.910   {\VBAR}         {\VBAR}         {\VBAR} 
   Diff (T-C)   {\VBAR} 0.010   {\VBAR} 0.002   {\VBAR} 4.50    {\VBAR} 0.000***
After           {\VBAR}         {\VBAR}         {\VBAR}         {\VBAR} 
   Control      {\VBAR} 0.917   {\VBAR}         {\VBAR}         {\VBAR} 
   Treated      {\VBAR} 0.920   {\VBAR}         {\VBAR}         {\VBAR} 
   Diff (T-C)   {\VBAR} 0.003   {\VBAR} 0.002   {\VBAR} 1.90    {\VBAR} 0.057*
                {\VBAR}         {\VBAR}         {\VBAR}         {\VBAR} 
Diff-in-Diff    {\VBAR} -0.006  {\VBAR} 0.003   {\VBAR} 2.29    {\VBAR} 0.022**
\HLI{56}
R-square:    0.01
* Means and Standard Errors are estimated by linear regression
**Inference: *** p<0.01; ** p<0.05; * p<0.1
{\smallskip}
. diff everschool, t(treated) p(time) cov( edad hhsize educfather educmother nodad nomom  sex) rep
> ort
{\bftt{{\underbar{DIFFERENCE-IN-DIFFERENCES WITH COVARIATES}}}}
{\smallskip}
{\bftt{{\underbar{DIFFERENCE-IN-DIFFERENCES ESTIMATION RESULTS}}}}
Number of observations in the DIFF-IN-DIFF: 58679
            Before         After    
   Control: 9363           13986       23349
   Treated: 14162          21168       35330
            23525          35154
Report - Covariates and coefficients:
\HLI{67}
 Variable(s)         {\VBAR}   Coeff.   {\VBAR} Std. Err. {\VBAR}    t    {\VBAR}  P>|t|
\HLI{21}{\PLUS}\HLI{12}{\PLUS}\HLI{11}{\PLUS}\HLI{9}{\PLUS}\HLI{10}
edad                 {\VBAR} 0.002      {\VBAR} 0.000     {\VBAR} 9.709   {\VBAR} 0.000
hhsize               {\VBAR} 0.002      {\VBAR} 0.000     {\VBAR} 5.964   {\VBAR} 0.000
educfather           {\VBAR} 0.004      {\VBAR} 0.000     {\VBAR} 11.947  {\VBAR} 0.000
educmother           {\VBAR} 0.004      {\VBAR} 0.000     {\VBAR} 13.769  {\VBAR} 0.000
nodad                {\VBAR} 0.000      {\VBAR} 0.000     {\VBAR}     .   {\VBAR}     .
nomom                {\VBAR} 0.000      {\VBAR} 0.000     {\VBAR}     .   {\VBAR}     .
sex                  {\VBAR} -0.004     {\VBAR} 0.001     {\VBAR} -2.968  {\VBAR} 0.003
\HLI{67}
\HLI{56}
 Outcome var.   {\VBAR} evers{\tytilde}l {\VBAR} S. Err. {\VBAR}   |t|   {\VBAR}  P>|t|
\HLI{16}{\PLUS}\HLI{9}{\PLUS}\HLI{9}{\PLUS}\HLI{9}{\PLUS}\HLI{9}
Before          {\VBAR}         {\VBAR}         {\VBAR}         {\VBAR} 
   Control      {\VBAR} 0.900   {\VBAR}         {\VBAR}         {\VBAR} 
   Treated      {\VBAR} 0.910   {\VBAR}         {\VBAR}         {\VBAR} 
   Diff (T-C)   {\VBAR} 0.010   {\VBAR} 0.002   {\VBAR} 4.50    {\VBAR} 0.000***
After           {\VBAR}         {\VBAR}         {\VBAR}         {\VBAR} 
   Control      {\VBAR} 0.917   {\VBAR}         {\VBAR}         {\VBAR} 
   Treated      {\VBAR} 0.920   {\VBAR}         {\VBAR}         {\VBAR} 
   Diff (T-C)   {\VBAR} 0.003   {\VBAR} 0.002   {\VBAR} 1.90    {\VBAR} 0.057*
                {\VBAR}         {\VBAR}         {\VBAR}         {\VBAR} 
Diff-in-Diff    {\VBAR} -0.006  {\VBAR} 0.003   {\VBAR} 2.29    {\VBAR} 0.022**
\HLI{56}
R-square:    0.01
* Means and Standard Errors are estimated by linear regression
**Inference: *** p<0.01; ** p<0.05; * p<0.1
{\smallskip}
. diff everschool, t(treated) p(time) cov( edad hhsize educfather educmother nodad nomom  sex) rep
> ort bs
{\bftt{{\underbar{DIFFERENCE-IN-DIFFERENCES WITH COVARIATES}}}}
(running {\bftt{regress}} on estimation sample)
{\smallskip}
Bootstrap replications (50)
\HLI{4}{\PLUS}\HLI{3} 1 \HLI{3}{\PLUS}\HLI{3} 2 \HLI{3}{\PLUS}\HLI{3} 3 \HLI{3}{\PLUS}\HLI{3} 4 \HLI{3}{\PLUS}\HLI{3} 5 
..................................................    50
{\smallskip}
{\bftt{{\underbar{DIFFERENCE-IN-DIFFERENCES ESTIMATION RESULTS}}}}
Number of observations in the DIFF-IN-DIFF: 58679
            Before         After    
   Control: 9363           13986       23349
   Treated: 14162          21168       35330
            23525          35154
Report - Covariates and coefficients:
\HLI{67}
 Variable(s)         {\VBAR}   Coeff.   {\VBAR} Std. Err. {\VBAR}    z    {\VBAR}  P>|z|
\HLI{21}{\PLUS}\HLI{12}{\PLUS}\HLI{11}{\PLUS}\HLI{9}{\PLUS}\HLI{10}
edad                 {\VBAR} 0.002      {\VBAR} 0.000     {\VBAR} 8.980   {\VBAR} 0.000
hhsize               {\VBAR} 0.002      {\VBAR} 0.000     {\VBAR} 6.974   {\VBAR} 0.000
educfather           {\VBAR} 0.004      {\VBAR} 0.000     {\VBAR} 11.794  {\VBAR} 0.000
educmother           {\VBAR} 0.004      {\VBAR} 0.000     {\VBAR} 16.688  {\VBAR} 0.000
nodad                {\VBAR} 0.000      {\VBAR} 0.000     {\VBAR}     .   {\VBAR}     .
nomom                {\VBAR} 0.000      {\VBAR} 0.000     {\VBAR}     .   {\VBAR}     .
sex                  {\VBAR} -0.004     {\VBAR} 0.001     {\VBAR} -2.913  {\VBAR} 0.004
\HLI{67}
Bootstrapped Standard Errors
{\smallskip}
\HLI{56}
 Outcome var.   {\VBAR} evers{\tytilde}l {\VBAR} S. Err. {\VBAR}   |t|   {\VBAR}  P>|t|
\HLI{16}{\PLUS}\HLI{9}{\PLUS}\HLI{9}{\PLUS}\HLI{9}{\PLUS}\HLI{9}
Before          {\VBAR}         {\VBAR}         {\VBAR}         {\VBAR} 
   Control      {\VBAR} 0.900   {\VBAR}         {\VBAR}         {\VBAR} 
   Treated      {\VBAR} 0.910   {\VBAR}         {\VBAR}         {\VBAR} 
   Diff (T-C)   {\VBAR} 0.010   {\VBAR} 0.002   {\VBAR} 4.46    {\VBAR} 0.000***
After           {\VBAR}         {\VBAR}         {\VBAR}         {\VBAR} 
   Control      {\VBAR} 0.917   {\VBAR}         {\VBAR}         {\VBAR} 
   Treated      {\VBAR} 0.920   {\VBAR}         {\VBAR}         {\VBAR} 
   Diff (T-C)   {\VBAR} 0.003   {\VBAR} 0.001   {\VBAR} 2.72    {\VBAR} 0.007***
                {\VBAR}         {\VBAR}         {\VBAR}         {\VBAR} 
Diff-in-Diff    {\VBAR} -0.006  {\VBAR} 0.003   {\VBAR} 2.29    {\VBAR} 0.022**
\HLI{56}
R-square:    0.01
* Means and Standard Errors are estimated by linear regression
**Inference: *** p<0.01; ** p<0.05; * p<0.1
{\smallskip}
. 
. 
. 
\end{stlog}



\section{PROPENSITY SCORE MATCHING }

CANASTA es un programa de apoyo alimentario que buscaba proporcionar una canasta básica de alimentos a familias en situación de vulnerabilidad y pobreza. El objetivo principal del programa es mejorar la nutrición y la alimentación de los hogares más necesitados, contribuyendo así a mejorar su calidad de vida. 
\vspace{0.5cm}

El programa está dirigido a familias de escasos recursos, en especial a aquellas que se encontraban en situación de pobreza o pobreza extrema, y cuyo acceso a una alimentación adecuada era limitado. La selección de beneficiarios se basaba en criterios socioeconómicos establecidos por las autoridades estatales. Por tanto, la asignación del programa no es aleatoria. 
En este ejercicio se busca estimar el efecto del programa Canasta sobre la estatura según la edad de los niños beneficiarios. En ese sentido, los investigadores tienen diferentes herramientas para balancear la muestra, de modo que los grupos sean comparables. Una de esas estrategias es Propensity Score Matching (PSM).
\vspace{0.5cm}

En la base de datos "emparejmaiento\_base" encontrará información de una muestra de 4 mil niños entre tratados y no tratados. Se le pide implementar el PSM siguiendo los siguientes pasos: 

a) Explore la base de datos identificando la variable tratamiento, la variable de interés y las variables que pueden determinar si el niño accede o no al programa. 

b) Estime la probabilidad predicha de participación en el programa para cada individuo. 

c) Realice el emparejamiento con el vecino más cercano sin reemplazo y estime el efecto del programa. 

d) Realice el emparejamiento con el vecino más cercano con reemplazo y estime el efecto del programa. 

e) Realice el emparejamiento con los 5 vecinos más cercano sin reemplazo y estime el efecto del programa. 

f) Realice el emparejamiento con los 5 vecinos más cercano con reemplazo y estime el efecto del programa. 

g) Realice el emparejamiento con kernel y estime el efecto del programa. h) Compare la calidad del emparejamiento de los métodos aplicados. ¿Cuál realiza el mejor emparejamiento? 

i) Compare los efectos estimados de los diferentes métodos. ¿Cambian mucho? j) ¿Cuál de los métodos elegiría usted para presentar en una investigación? ¿Por qué? 






\bibliographystyle{apacite}
\bibliography{scholar}
\end{document}
